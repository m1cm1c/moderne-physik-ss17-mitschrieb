\chapter{Übung 2}

\section*{Aufgabe 1}

\begin{description}
	\item[a)] $f'(x) = x^3 + 2x - 5 + \frac{1}{x}$
	\item[b)] $f(x) = \cos(x) + \sin(x) \tan(x) = \frac{\cos^2(x)}{\cos(x)} + \frac{\sin^2(x)}{\cos(x)} = \frac{1}{\cos(x)}$, also $f'(x) =\sin(x) \frac{1}{\cos^2(x)} = \frac{\sin(x)}{\cos^2(x)} = \frac{\tan(x)}{\cos(x)}$
	\item[c)] $f(x) = \frac{1}{2 \sqrt{2}} \arctan\left( \frac{2 \sqrt{2}}{1 - x^2} \right)$
	
	Bronstein 6.1.22: $g(x) = \arctan(x)$ $\Rightarrow$ $g'(x) = \frac{1}{1 + x^2}$
	
	Damit: $f'(x) 
	= \frac{1}{2 \sqrt{x}} \frac{1}{1 + \frac{2x^2}{(1 - x^2)^2}} \sqrt{2} \frac{1 - x^2 + 2x^2}{(1 - x^2)^2}
	= \frac{1}{2} \frac{1 + x^2}{(1 - x)^2 + 2x^2}
	= \frac{1}{2} \frac{1 + x^2}{x^4 + 1}$
	
	\item[d)] $f(x) = \frac{1}{4 \sqrt{2}} \log \left( \frac{x^2 + \sqrt{2} x + 1}{x^2 - \sqrt{2}x + 1} \right)$
	
	$f'(x) 
	= \frac{1}{4 \sqrt{2}} \left( \frac{x^2 + \sqrt{2}x + 1}{x^2 - \sqrt{2}x + 1} \right)^{-1} \frac{(2x + \sqrt{2})(x^2 - \sqrt{2}x + 1)-(2x + \sqrt{2})(x^2 + \sqrt{2}x + 1)}{(x^2 + \sqrt{2}x + 1)^2}
	= \dots 
	= \frac{1}{2} \frac{1 - x^2}{x^4 + 1}$
\end{description}

\section*{Aufgabe 2}

\begin{description}
	\item[a)] $f(x) = \cos(x)$, $f'(x) = -\sin(x)$, $f''(x) = -\cos(x)$, also $f(0) = 1$, $f'(0) = 0$, $f''(0) = -1$.
	
	Damit ergibt sich: $f(x) = 1 + 0 x + \left( -\frac{1}{2} \right)x^2 + \mathcal{O}(x^3) \approx 1 - \frac{x^2}{2}$
	
	\item[b)] $f(x) = \log(1 - x)$, $f'(x) = -\frac{1}{1 - x} = \frac{1}{x - 1}$, $f''(x) = -\frac{1}{(x - 1)^2}$, also $f(0) = 0$, $f'(0) = 1$, $f''(0) = -1$.
	
	Damit ergibt sich: $f(x) = -x - \frac{x^2}{2} + \mathcal{O}(x^3)$.
\end{description}

\section*{Aufgabe 3}

\begin{description}
	\item[a)] $f(x) = e^{\lambda x} \sin(3x)$, man rechne:
	\begin{align*}
		\int e^{\lambda x} \sin(3x) \mathrm{d} x	
		&= \frac{1}{\lambda} e^{\lambda x} \sin(3x) - \frac{3}{\lambda} \int e^{\lambda x} \cos(3x) \mathrm{d} x \\
		&= \frac{1}{\lambda} e^{\lambda x} \sin(3x) - \frac{3}{\lambda} \left[ \frac{1}{\lambda} e^{\lambda x} \cos(3x) + \frac{3}{\lambda} \int e^{\lambda x} \sin(3x) \mathrm{d} x \right]
	\end{align*}
	
	Damit ergibt sich:
	\[
		\left( 1 + \frac{9}{\lambda^2} \right) \int e^{\lambda x} \sin(3x) \mathrm{d} x = \frac{1}{\lambda^2} e^{\lambda x} \left( \lambda \sin(3x) - 3 \cos(3x) \right)	
	\]
	
	Noch durch den Vorfaktor teilen:
	\[
		\int e^{\lambda x} \sin(3x) \mathrm{d} x 
		= \frac{1}{\lambda^2 + 9} e^{\lambda x} \left( \lambda \sin(3x) - 3 \cos(3x) \right) + C
	\]

	\item[b)] Stichwort: Partialbruchzerlegung
	\[ 
		\frac{1}{(1 - x)(1 + x)} 
		= \frac{A}{1 - x} + \frac{B}{1 + x} 
		= \frac{A + Ax + B - Bx}{(1 - x)(1 + x)} 
		= \frac{x(A - B) + (A + B)}{(x - 1)(x + 1)}
	\]
	Vergleich: $A = \frac{1}{2}$, $B = \frac{1}{2}$
	
	Nun einfach:
	\begin{align*}
		\int \frac{1}{(x - 1)(x + 1)} \mathrm{d} x 
		&= \int \frac{1}{2} \frac{1}{x - 1} + \frac{1}{2} \frac{1}{1 + x} \mathrm{d} x 
		= - \frac{1}{2} \log(1 - x) + \frac{1}{2} \log(1 + x) + C \\
		&= \frac{1}{2} \log \left( \frac{1 + x}{1 - x} \right) + C
	\end{align*}

	\item[b, i)] Verwende Substitution $x = \sin(z)$, $\frac{\mathrm{d} x}{\mathrm{d} z} = \cos(z)$, $z = \arcsin(x)$
	
	\begin{align*}
		\int_0^1 \frac{\mathrm{d} x}{\sqrt{1 - x^2}}	 
		&= \int^{\pi/2}_{0} \frac{1}{\sqrt{1 - \sin^2(z)}} \frac{\mathrm{d} x}{\mathrm{d} z} \mathrm{d} z
		= \int^{\pi / 2}_0 \frac{1}{\cos(z)} \cos(z) \mathrm{d} z
		= \int^{\pi / 2}_0 1 \mathrm{d} z \\
		&= \left[ z \right]^{\pi / 2}_0 = \frac{\pi}{2}
	\end{align*}

	\item[b, ii)]  $F(a, n) = \int^\infty_{-\infty} x^{2n} e^{-ax^2} \mathrm{d} x$ mit $a > 0$
	\begin{align*}
			F(a, n) 
			&= \left( \frac{\mathrm{d}}{\mathrm{d} a} \right)^n (-1)^n \int^{\infty}_{-\infty} e^{-ax^2} \mathrm{d} x
			= \left( \frac{\mathrm{d}}{\mathrm{d} a} \right)^n (-1)^n \sqrt{\frac{\pi}{a}} \\
			&=  \sqrt{\pi} \frac{\mathrm{d}^n}{\mathrm{d} a^n} \left( a^{-1/2} \right) (-1)^n
			= \sqrt{\pi} (-1)^n \frac{\mathrm{d}^{n -1}}{\mathrm{d} a^{n - 1}} \left( -\frac{1}{2} a^{-3/2} \right) \\
			&= \sqrt{\pi} (-1)^n \frac{\mathrm{d}^{n - 2}}{\mathrm{d} a^{n - 2}} \left( \frac{1}{2} \frac{3}{2} a^{-5/2} \right) 
			= \dots 
			= \sqrt{\pi} (-1)^n (-1)^n \left( \frac{1}{2} \frac{3}{2} \cdot \hdots \cdot \frac{2n - 1}{2} a^{-(2n + 1)/2} \right) \\
			&= \dots
	\end{align*}
\end{description}

\section*{Aufgabe 4}

\begin{description}
	\item[a)] 
	\begin{tabular}{|c||c|c|c|c|c|c|c|c|}
		\hline 
		$z = x + iy$ & $1 + 1i$ & $3 - 4i$ & $-3 + 2i$ & -2 & $5 - 12 i$ & $-1 - i$ & 1 + 1.7i \\
		\hline
		$r = \mabs{z}$ & $\sqrt{2}$ & $5$ & $\sqrt{13}$ & $2$ & $13$ & $\sqrt{2}$ & 2 \\
		\hline 
		$\phi = \arg(z)$ & $\frac{\pi}{4}$ & $1.705 \pi$ & $0.813 \pi$ & $\pi$ & $1.626 \pi$ & $\frac{5}{4} \pi$ & $-\pi/3$ \\
		\hline 
	\end{tabular}
 
	\item[b, i)] $z = \frac{1 + i}{2 + 3i} = \frac{(1 + i)(2 - 3i)}{(2 + 3i)(2 - 3i)} = \dots = \frac{5 - i}{13} \approx \frac{1}{13} \sqrt{26} e^{1.937 \pi i}$
	
	\item[b, ii)] $z = \frac{1}{\sqrt{1 + i}} = (1 + i)^{-1/2} = \left( \sqrt{2}\exp^{i \frac{\pi}{4}} \right)^{-1/2}$
	
		$z'_1 = 2^{-1/4} e^{-i \frac{\pi}{8} + 2 \pi i}$
		
		$z'_2 = 2^{-1/4} e^{-i \frac{9}{8} \pi + 2 \pi i}$
\end{description}

