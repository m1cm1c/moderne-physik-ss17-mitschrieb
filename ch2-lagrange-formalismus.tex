\chapter{Lagrange-Formalismus}
Aus der Lagrange-Funktion folgt:

\[ \sum_{j=1}^s [ \dfrac{\dint}{\dint t} \dfrac{\partial L}{\partial \dot q_j} - \dfrac{\partial L}{\partial q_i} ] \delta q_i = 0 \]

Damit folgt die Lagrange-Gleichung 2. Art:

\[ \dfrac{\dint}{\dint t} \dfrac{\partial L}{\partial \dot q_j} - \dfrac{\partial L}{\partial q_j} = 0; j = 1, \ldots, s\]


\section{Bemerkungen}
\begin{itemize}
    \item Die Zwangskräfte wurden eliminiert.
    \item Hier sind Energie und Arbeit zentrale Begriffe. Vgl. Kraft und Impuls bei Newton.
    \item Lagrangegleichungen sind invariant unter Punkttransformation.
\end{itemize}


\section{Zyklische Koordinaten}
$q_i$ ist eine {\bf zyklische Koordinate}, gdw. die Lagrange-Funktion nicht explizit von $q_i$ abhängt. Die Lagrange-Funktion kann dabei aber dennoch von $\dot q_i$ abhängen.

\[ q_i \text{ ist zyklische Koordinate} \Leftrightarrow \pdv{L}{q_i} = 0 \Leftrightarrow \pdv{L}{\dot q_i} = \text{const.} \]

\[ \dv{}{t} \pdv{L}{\dot q_i} = 0 \Leftrightarrow \pdv{L}{\dot q_i} = \text{const.} = p_i \]

Dabei ist $\pdv{L}{\dot q_i} = p_i$ der {\bf verallgemeinerte Impuls}.

Zyklische Koordinaten führen immer sofort zu einem {\bf Erhaltungssatz}. Möglichst viele generalisierte Koordinaten sollten zyklisch sein.