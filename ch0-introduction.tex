\chapter{Einleitung}

Verstehen im Gegensatz zur "`klassischen Physik"', entwickelt ab Anfang des 20. Jhd. Wichtigste Entwicklungen im 1. Viertel des Jhd.

Bis dahin (grob vereinfacht):
\begin{itemize}
	\item Newtonsche Mechanik
	\item Maxwells Elektrodynamik
\end{itemize}

\textbf{Paradigma}: alles im Prinzip berechenbar (Zeitentwicklung von System), solange die Anfangsbedingungen (alle $\vec{x}_i(t_0)$, $\dot{\vec{x}}_i(t_0)$, wobei $i =$ alle Punkte) bekannt sind.

\textbf{Aber}: Experimente zeigen immer mehr Widersprüche, zum Beispiel:
\begin{itemize}
	\item Michelson-Morley-Experiment: es gibt keinen "`Äther"', daraus hat sich die Spezielle Relativitätstheorie von Einstein (SRT) entwickelt
	\item Diskrete Energiespektren (Spektrallinien)
	\item Welleneigenschaft von Teilchen
	\item Teilcheneigenschaften von Lichtwellen
	\item Schwarzkörperstrahlung
	\item[$\Rightarrow$] das führte zur Quantenphysik 
\end{itemize}

Aufbau der Vorlesung:
\begin{enumerate}
	\item Klassische Mechanik
	\begin{itemize}
		\item Newtonsche Mechanik (sehr kurz)
		\begin{itemize}
			\item Entwicklung der formaleren analytischen Mechanik (Lagrange, Hamilton, Jacobi)
			\item erlaubt theoretische Diskussionen, z.B. Symmetrien und Erhaltungssätze; andere Konzepte, die in der Quantenmechanik (QM) eine Rolle spielen; "`Hamiltonoperator"'; kanonisch konjugierte Variable
		\end{itemize}
	\end{itemize}
	
	\item Relativität
	\begin{itemize}
		\item Symmetrie von Raum und Zeit
		\begin{itemize}
			\item daraus die Spezielle Relativität entwickeln (etwas losgelöst von klassischer Mechanik)
			\item formale Entwicklung, dann radikale Konsequenzen bestimmen
		\end{itemize}
	\end{itemize}
	
	\item Quantenmechanik
	\begin{itemize}
		\item wenig (?) Historie
		\item einfache eindimensionale Probleme 
		\begin{itemize}
			\item Schrödingergleichung
			\item Zusammenhang zur Wellenmechanik
		\end{itemize}
		\item Postulate der Quantenmechanik
		\item Symmetrien (?)
		\item Wasserstoffatom (System der Elemente)
		\item Identische Teilchen 
		\item Beispiele \dots
	\end{itemize}
\end{enumerate}